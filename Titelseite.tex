\begin{titlepage}
        \centering
        \vspace*{2.5cm}
        {\huge{Grundpraktikum}}\\
        \vspace*{1cm}
        {\huge{\textbf{\thetitle}}}\\
        \vspace{2cm}
        {\Large{\theauthor}}\\
        {\Large{\thedate}}\\
        \vspace{2cm}
        \begin{tabular}{ll}
				\textbf{Student:} & Name (Matrikelnr) \\
				& Mail-Adresse\\
				\textbf{Student:} & Name (Matrikelnr)  \\
				& Mail-Adresse\\ 
				\textbf{Betreuer:} & Name \\ 
				& Mail-Adresse\\
				\textbf{Raum:} &  \\ 
				\textbf{Messplatz:} &  \\
		\end{tabular}
        \vspace{1cm}
        \begin{abstract}
            \noindent Das Template kann gerne kopiert werden. Wichtig ist dabei, dass man den Settings-Ordner mitnimmt, damit das \TeX-Dokument funktioniert. Es wird davon abgeraden, die Inhalte vom GPR1/GPR2 zu kopieren (wegen Plagiard). Zudem sollten ein, zwei Ergebnisse aus der Analyse hier in den Abstract (Zusammenfassung mit eingebunden werden.
            \begin{table}[H]
                \centering
                \begin{tabular}{c|c}
                   Ergebnis 1  & Ergebnis 2 \\
                   \hline
                    x1 & x2
                \end{tabular}
                % \caption{Caption}
                % \label{tab:my_label}
            \end{table}
        \end{abstract}
    \end{titlepage}